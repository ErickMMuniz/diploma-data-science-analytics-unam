% Options for packages loaded elsewhere
\PassOptionsToPackage{unicode}{hyperref}
\PassOptionsToPackage{hyphens}{url}
\documentclass[
]{article}
\usepackage{xcolor}
\usepackage[margin=1in]{geometry}
\usepackage{amsmath,amssymb}
\setcounter{secnumdepth}{-\maxdimen} % remove section numbering
\usepackage{iftex}
\ifPDFTeX
  \usepackage[T1]{fontenc}
  \usepackage[utf8]{inputenc}
  \usepackage{textcomp} % provide euro and other symbols
\else % if luatex or xetex
  \usepackage{unicode-math} % this also loads fontspec
  \defaultfontfeatures{Scale=MatchLowercase}
  \defaultfontfeatures[\rmfamily]{Ligatures=TeX,Scale=1}
\fi
\usepackage{lmodern}
\ifPDFTeX\else
  % xetex/luatex font selection
\fi
% Use upquote if available, for straight quotes in verbatim environments
\IfFileExists{upquote.sty}{\usepackage{upquote}}{}
\IfFileExists{microtype.sty}{% use microtype if available
  \usepackage[]{microtype}
  \UseMicrotypeSet[protrusion]{basicmath} % disable protrusion for tt fonts
}{}
\makeatletter
\@ifundefined{KOMAClassName}{% if non-KOMA class
  \IfFileExists{parskip.sty}{%
    \usepackage{parskip}
  }{% else
    \setlength{\parindent}{0pt}
    \setlength{\parskip}{6pt plus 2pt minus 1pt}}
}{% if KOMA class
  \KOMAoptions{parskip=half}}
\makeatother
\usepackage{color}
\usepackage{fancyvrb}
\newcommand{\VerbBar}{|}
\newcommand{\VERB}{\Verb[commandchars=\\\{\}]}
\DefineVerbatimEnvironment{Highlighting}{Verbatim}{commandchars=\\\{\}}
% Add ',fontsize=\small' for more characters per line
\usepackage{framed}
\definecolor{shadecolor}{RGB}{248,248,248}
\newenvironment{Shaded}{\begin{snugshade}}{\end{snugshade}}
\newcommand{\AlertTok}[1]{\textcolor[rgb]{0.94,0.16,0.16}{#1}}
\newcommand{\AnnotationTok}[1]{\textcolor[rgb]{0.56,0.35,0.01}{\textbf{\textit{#1}}}}
\newcommand{\AttributeTok}[1]{\textcolor[rgb]{0.13,0.29,0.53}{#1}}
\newcommand{\BaseNTok}[1]{\textcolor[rgb]{0.00,0.00,0.81}{#1}}
\newcommand{\BuiltInTok}[1]{#1}
\newcommand{\CharTok}[1]{\textcolor[rgb]{0.31,0.60,0.02}{#1}}
\newcommand{\CommentTok}[1]{\textcolor[rgb]{0.56,0.35,0.01}{\textit{#1}}}
\newcommand{\CommentVarTok}[1]{\textcolor[rgb]{0.56,0.35,0.01}{\textbf{\textit{#1}}}}
\newcommand{\ConstantTok}[1]{\textcolor[rgb]{0.56,0.35,0.01}{#1}}
\newcommand{\ControlFlowTok}[1]{\textcolor[rgb]{0.13,0.29,0.53}{\textbf{#1}}}
\newcommand{\DataTypeTok}[1]{\textcolor[rgb]{0.13,0.29,0.53}{#1}}
\newcommand{\DecValTok}[1]{\textcolor[rgb]{0.00,0.00,0.81}{#1}}
\newcommand{\DocumentationTok}[1]{\textcolor[rgb]{0.56,0.35,0.01}{\textbf{\textit{#1}}}}
\newcommand{\ErrorTok}[1]{\textcolor[rgb]{0.64,0.00,0.00}{\textbf{#1}}}
\newcommand{\ExtensionTok}[1]{#1}
\newcommand{\FloatTok}[1]{\textcolor[rgb]{0.00,0.00,0.81}{#1}}
\newcommand{\FunctionTok}[1]{\textcolor[rgb]{0.13,0.29,0.53}{\textbf{#1}}}
\newcommand{\ImportTok}[1]{#1}
\newcommand{\InformationTok}[1]{\textcolor[rgb]{0.56,0.35,0.01}{\textbf{\textit{#1}}}}
\newcommand{\KeywordTok}[1]{\textcolor[rgb]{0.13,0.29,0.53}{\textbf{#1}}}
\newcommand{\NormalTok}[1]{#1}
\newcommand{\OperatorTok}[1]{\textcolor[rgb]{0.81,0.36,0.00}{\textbf{#1}}}
\newcommand{\OtherTok}[1]{\textcolor[rgb]{0.56,0.35,0.01}{#1}}
\newcommand{\PreprocessorTok}[1]{\textcolor[rgb]{0.56,0.35,0.01}{\textit{#1}}}
\newcommand{\RegionMarkerTok}[1]{#1}
\newcommand{\SpecialCharTok}[1]{\textcolor[rgb]{0.81,0.36,0.00}{\textbf{#1}}}
\newcommand{\SpecialStringTok}[1]{\textcolor[rgb]{0.31,0.60,0.02}{#1}}
\newcommand{\StringTok}[1]{\textcolor[rgb]{0.31,0.60,0.02}{#1}}
\newcommand{\VariableTok}[1]{\textcolor[rgb]{0.00,0.00,0.00}{#1}}
\newcommand{\VerbatimStringTok}[1]{\textcolor[rgb]{0.31,0.60,0.02}{#1}}
\newcommand{\WarningTok}[1]{\textcolor[rgb]{0.56,0.35,0.01}{\textbf{\textit{#1}}}}
\usepackage{graphicx}
\makeatletter
\newsavebox\pandoc@box
\newcommand*\pandocbounded[1]{% scales image to fit in text height/width
  \sbox\pandoc@box{#1}%
  \Gscale@div\@tempa{\textheight}{\dimexpr\ht\pandoc@box+\dp\pandoc@box\relax}%
  \Gscale@div\@tempb{\linewidth}{\wd\pandoc@box}%
  \ifdim\@tempb\p@<\@tempa\p@\let\@tempa\@tempb\fi% select the smaller of both
  \ifdim\@tempa\p@<\p@\scalebox{\@tempa}{\usebox\pandoc@box}%
  \else\usebox{\pandoc@box}%
  \fi%
}
% Set default figure placement to htbp
\def\fps@figure{htbp}
\makeatother
\setlength{\emergencystretch}{3em} % prevent overfull lines
\providecommand{\tightlist}{%
  \setlength{\itemsep}{0pt}\setlength{\parskip}{0pt}}
\usepackage{bookmark}
\IfFileExists{xurl.sty}{\usepackage{xurl}}{} % add URL line breaks if available
\urlstyle{same}
\hypersetup{
  pdftitle={K means},
  pdfauthor={Luis Daniel López Muñoz},
  hidelinks,
  pdfcreator={LaTeX via pandoc}}

\title{K means}
\author{Luis Daniel López Muñoz}
\date{}

\begin{document}
\maketitle

\begin{Shaded}
\begin{Highlighting}[]
\FunctionTok{library}\NormalTok{(tidyverse)}
\end{Highlighting}
\end{Shaded}

\begin{verbatim}
## -- Attaching core tidyverse packages ------------------------ tidyverse 2.0.0 --
## v dplyr     1.1.4     v readr     2.1.5
## v forcats   1.0.0     v stringr   1.5.1
## v ggplot2   3.5.2     v tibble    3.3.0
## v lubridate 1.9.4     v tidyr     1.3.1
## v purrr     1.1.0     
## -- Conflicts ------------------------------------------ tidyverse_conflicts() --
## x dplyr::filter() masks stats::filter()
## x dplyr::lag()    masks stats::lag()
## i Use the conflicted package (<http://conflicted.r-lib.org/>) to force all conflicts to become errors
\end{verbatim}

\begin{Shaded}
\begin{Highlighting}[]
\FunctionTok{library}\NormalTok{(janitor)}
\end{Highlighting}
\end{Shaded}

\begin{verbatim}
## 
## Attaching package: 'janitor'
## 
## The following objects are masked from 'package:stats':
## 
##     chisq.test, fisher.test
\end{verbatim}

\begin{Shaded}
\begin{Highlighting}[]
\FunctionTok{library}\NormalTok{(skimr)}
\FunctionTok{library}\NormalTok{(glue)}
\FunctionTok{library}\NormalTok{(readr)}
\FunctionTok{library}\NormalTok{(recipes)}
\end{Highlighting}
\end{Shaded}

\begin{verbatim}
## 
## Attaching package: 'recipes'
## 
## The following object is masked from 'package:stringr':
## 
##     fixed
## 
## The following object is masked from 'package:stats':
## 
##     step
\end{verbatim}

\begin{Shaded}
\begin{Highlighting}[]
\FunctionTok{library}\NormalTok{(cluster)}
\FunctionTok{library}\NormalTok{(factoextra)}
\end{Highlighting}
\end{Shaded}

\begin{verbatim}
## Welcome! Want to learn more? See two factoextra-related books at https://goo.gl/ve3WBa
\end{verbatim}

Este documento realiza \textbf{aprendizaje no supervisado (K-means)}
sobre el conjunto \texttt{habits.csv} (1,000 estudiantes) para
identificar \textbf{perfiles de hábitos} y relacionarlos con el
\textbf{exam score}. El flujo será: \emph{exploración → preprocesamiento
→ selección de k → K-means → interpretación}.

Leemos los datos

\begin{Shaded}
\begin{Highlighting}[]
\NormalTok{Datos}\OtherTok{\textless{}{-}}\FunctionTok{read.csv}\NormalTok{(ruta\_base)}
\end{Highlighting}
\end{Shaded}

Comenzaremos por ver si nuestros datos contienen variables con valor
``NA''.

\begin{Shaded}
\begin{Highlighting}[]
\CommentTok{\# Conteo de NA por columna}
\NormalTok{na\_profile }\OtherTok{\textless{}{-}}\NormalTok{ Datos }\SpecialCharTok{\%\textgreater{}\%}
  \FunctionTok{summarise}\NormalTok{(}\FunctionTok{across}\NormalTok{(}\FunctionTok{everything}\NormalTok{(), }\SpecialCharTok{\textasciitilde{}}\FunctionTok{sum}\NormalTok{(}\FunctionTok{is.na}\NormalTok{(.)))) }\SpecialCharTok{\%\textgreater{}\%}
  \FunctionTok{pivot\_longer}\NormalTok{(}\FunctionTok{everything}\NormalTok{(), }\AttributeTok{names\_to =} \StringTok{"variable"}\NormalTok{, }\AttributeTok{values\_to =} \StringTok{"na\_count"}\NormalTok{) }\SpecialCharTok{\%\textgreater{}\%}
  \FunctionTok{arrange}\NormalTok{(}\FunctionTok{desc}\NormalTok{(na\_count))}

\NormalTok{na\_profile }\SpecialCharTok{\%\textgreater{}\%} \FunctionTok{filter}\NormalTok{(na\_count }\SpecialCharTok{\textgreater{}} \DecValTok{0}\NormalTok{)}
\end{Highlighting}
\end{Shaded}

\begin{verbatim}
## # A tibble: 0 x 2
## # i 2 variables: variable <chr>, na_count <int>
\end{verbatim}

De lo cual, obtenemos que no tenemos datos faltantes.

Ahora veremos que tipo de datos constituyen cada columna:

\begin{Shaded}
\begin{Highlighting}[]
\NormalTok{type\_profile }\OtherTok{\textless{}{-}}\NormalTok{ Datos }\SpecialCharTok{\%\textgreater{}\%}
  \FunctionTok{summarise}\NormalTok{(}\FunctionTok{across}\NormalTok{(}\FunctionTok{everything}\NormalTok{(), }\SpecialCharTok{\textasciitilde{}}\FunctionTok{class}\NormalTok{(.x)[}\DecValTok{1}\NormalTok{])) }\SpecialCharTok{\%\textgreater{}\%}
  \FunctionTok{pivot\_longer}\NormalTok{(}\FunctionTok{everything}\NormalTok{(), }\AttributeTok{names\_to =} \StringTok{"variable"}\NormalTok{, }\AttributeTok{values\_to =} \StringTok{"class"}\NormalTok{)}

\NormalTok{type\_profile}
\end{Highlighting}
\end{Shaded}

\begin{verbatim}
## # A tibble: 15 x 2
##    variable                      class    
##    <chr>                         <chr>    
##  1 age                           integer  
##  2 gender                        character
##  3 study_hours_per_day           numeric  
##  4 social_media_hours            numeric  
##  5 netflix_hours                 numeric  
##  6 part_time_job                 character
##  7 attendance_percentage         numeric  
##  8 sleep_hours                   numeric  
##  9 diet_quality                  character
## 10 exercise_frequency            integer  
## 11 parental_education_level      character
## 12 internet_quality              character
## 13 mental_health_rating          integer  
## 14 extracurricular_participation character
## 15 exam_score                    numeric
\end{verbatim}

\subsection{Procesamiento de Datos}\label{procesamiento-de-datos}

Aunque el conjunto de datos incluye variables categóricas (por ejemplo,
\emph{género}, \emph{tipo de empleo parcial}, \emph{calidad de dieta} o
\emph{participación extracurricular}), estas no se incluyen en el
proceso de entrenamiento del algoritmo \textbf{K-means}.

El motivo es que \textbf{K-means se basa en distancias euclidianas}, las
cuales requieren variables numéricas y continuas. Si las variables
categóricas se transformaran en indicadores binarios (\emph{one-hot
encoding}), podrían distorsionar los resultados, ya que la presencia o
ausencia de una categoría tendría el mismo peso que cambios en variables
numéricas como las horas de estudio o de sueño.

En lugar de forzar su inclusión, se opta por un enfoque más sólido:\\
- \textbf{Entrenar el modelo solo con variables numéricas}, que son más
apropiadas para la métrica de distancia de K-means.\\
- \textbf{Utilizar las variables categóricas posteriormente para
interpretar los clusters}, analizando cómo se distribuyen estas
características en cada grupo identificado.

Este procedimiento permite generar clusters más robustos y, al mismo
tiempo, enriquecer su interpretación con la información cualitativa del
conjunto de datos.

\begin{Shaded}
\begin{Highlighting}[]
\CommentTok{\# Selección de variables numéricas para clustering}
\NormalTok{Datos\_num }\OtherTok{\textless{}{-}}\NormalTok{ Datos }\SpecialCharTok{\%\textgreater{}\%}
  \FunctionTok{select}\NormalTok{(age, study\_hours\_per\_day, social\_media\_hours, netflix\_hours,}
\NormalTok{         attendance\_percentage, sleep\_hours, exercise\_frequency,}
\NormalTok{         mental\_health\_rating, exam\_score)}

\CommentTok{\# Normalización (media = 0, desviación estándar = 1)}
\NormalTok{X\_scaled }\OtherTok{\textless{}{-}} \FunctionTok{scale}\NormalTok{(Datos\_num)}

\CommentTok{\# Resumen estadístico de las variables ya escaladas}
\FunctionTok{summary}\NormalTok{(}\FunctionTok{as.data.frame}\NormalTok{(X\_scaled))}
\end{Highlighting}
\end{Shaded}

\begin{verbatim}
##       age          study_hours_per_day social_media_hours  netflix_hours     
##  Min.   :-1.5155   Min.   :-2.41686    Min.   :-2.137028   Min.   :-1.69256  
##  1st Qu.:-0.7573   1st Qu.:-0.64682    1st Qu.:-0.687039   1st Qu.:-0.76243  
##  Median :-0.2158   Median :-0.03411    Median :-0.004691   Median :-0.01832  
##  Mean   : 0.0000   Mean   : 0.00000    Mean   : 0.000000   Mean   : 0.00000  
##  3rd Qu.: 1.0840   3rd Qu.: 0.64668    3rd Qu.: 0.677657   3rd Qu.: 0.65602  
##  Max.   : 1.5173   Max.   : 3.23367    Max.   : 4.004103   Max.   : 3.33015  
##  attendance_percentage  sleep_hours       exercise_frequency
##  Min.   :-2.99297      Min.   :-2.66647   Min.   :-1.50191  
##  1st Qu.:-0.65236      1st Qu.:-0.70949   1st Qu.:-1.00818  
##  Median : 0.02854      Median : 0.02438   Median :-0.02074  
##  Mean   : 0.00000      Mean   : 0.00000   Mean   : 0.00000  
##  3rd Qu.: 0.73339      3rd Qu.: 0.67671   3rd Qu.: 0.96671  
##  Max.   : 1.68825      Max.   : 2.87832   Max.   : 1.46044  
##  mental_health_rating   exam_score     
##  Min.   :-1.5586      Min.   :-3.0317  
##  1st Qu.:-0.8562      1st Qu.:-0.6588  
##  Median :-0.1538      Median : 0.0532  
##  Mean   : 0.0000      Mean   : 0.0000  
##  3rd Qu.: 0.8997      3rd Qu.: 0.6942  
##  Max.   : 1.6021      Max.   : 1.7999
\end{verbatim}

\subsection{Selección del número de
clusters}\label{selecciuxf3n-del-nuxfamero-de-clusters}

La elección adecuada de \textbf{k}, es decir, el número de clusters, es
un paso fundamental en el algoritmo \textbf{K-means}.\\
Uno de los métodos más utilizados es el \textbf{método del codo
(\emph{Elbow method})}, que evalúa cómo disminuye la \textbf{suma de
distancias al cuadrado dentro de los clusters (WSS, \emph{Within-Cluster
Sum of Squares})} a medida que aumenta el número de clusters.

La lógica es la siguiente:

\begin{itemize}
\tightlist
\item
  A medida que incrementamos \textbf{k}, la WSS siempre disminuye porque
  los clusters se vuelven más pequeños y homogéneos.\\
\item
  Sin embargo, llega un punto en el que la reducción adicional de WSS es
  marginal.\\
\item
  Ese punto de inflexión se denomina \textbf{``codo''}, y es el valor
  sugerido para \textbf{k}.
\end{itemize}

\begin{Shaded}
\begin{Highlighting}[]
\FunctionTok{fviz\_nbclust}\NormalTok{(X\_scaled, kmeans, }\AttributeTok{method =} \StringTok{"wss"}\NormalTok{) }\SpecialCharTok{+}
  \FunctionTok{labs}\NormalTok{(}\AttributeTok{title =} \StringTok{"Método del codo (Elbow method)"}\NormalTok{,}
       \AttributeTok{x =} \StringTok{"Número de clusters (k)"}\NormalTok{)}
\end{Highlighting}
\end{Shaded}

\begin{center}\includegraphics{tarea_2_files/figure-latex/elbow-1} \end{center}

En el gráfico del método del codo, la disminución de la \textbf{WSS} no
presenta un punto de inflexión claro.\\
La curva desciende de manera progresiva desde \(k = 1\) hasta
\(k = 10\), sin un cambio de pendiente lo suficientemente marcado como
para señalar un único valor de \(k\).

Cuando el ``codo'' no es evidente, el criterio del método se vuelve
ambiguo y es recomendable apoyarse en otros enfoques, como el
\textbf{índice de Silhouette}, que permite evaluar directamente la
calidad de la partición generada para cada número de clusters.

\subsubsection{3.1 Índice de Silhouette}\label{uxedndice-de-silhouette}

El \textbf{índice de Silhouette} evalúa simultáneamente la
\textbf{compacidad interna} de los clusters y su \textbf{separación}
respecto a los demás grupos.\\
Para cada observación, compara la distancia media a los puntos de su
propio cluster con la distancia media al cluster vecino más cercano.\\
El valor promedio por partición toma valores en \([-1, 1]\):\\
- cercano a \textbf{1} → clusters bien separados y compactos;\\
- cerca de \textbf{0} → fronteras difusas entre clusters;\\
- \textbf{negativo} → asignaciones potencialmente incorrectas.

Seleccionaremos el \(k\) cuyo \textbf{promedio de Silhouette} sea
\textbf{máximo}.

\begin{Shaded}
\begin{Highlighting}[]
\CommentTok{\# Gráfico automático del promedio de Silhouette por k}
\FunctionTok{fviz\_nbclust}\NormalTok{(X\_scaled, kmeans, }\AttributeTok{method =} \StringTok{"silhouette"}\NormalTok{) }\SpecialCharTok{+}
  \FunctionTok{labs}\NormalTok{(}\AttributeTok{title =} \StringTok{"Selección de k mediante índice de Silhouette"}\NormalTok{,}
       \AttributeTok{x =} \StringTok{"Número de clusters (k)"}\NormalTok{,}
       \AttributeTok{y =} \StringTok{"Silhouette promedio"}\NormalTok{)}
\end{Highlighting}
\end{Shaded}

\begin{center}\includegraphics{tarea_2_files/figure-latex/silhouette-1} \end{center}

El gráfico del índice de Silhouette muestra que el valor máximo se
alcanza en \textbf{k = 2}, con un promedio cercano a 0.125.\\
A partir de \textbf{k = 3}, los valores disminuyen y se estabilizan en
torno a 0.09--0.10, lo cual indica que dividir en más grupos no mejora
de manera significativa la separación ni la compacidad de los clusters.

Si bien los valores absolutos del índice no son elevados (lo cual
refleja que los clusters no están fuertemente diferenciados), el
criterio de Silhouette sugiere que la mejor partición de los datos se
obtiene con \textbf{dos clusters}.

\subsection{Ejecución del algoritmo
K-means}\label{ejecuciuxf3n-del-algoritmo-k-means}

Con base en el análisis previo, el número de clusters seleccionado es
\textbf{k = 2}, ya que fue el valor óptimo de acuerdo con el índice de
Silhouette.

En esta sección se ejecuta el algoritmo \textbf{K-means} sobre la matriz
de variables normalizadas y se generan las asignaciones de cluster para
cada estudiante.\\
Posteriormente, se presentan dos salidas clave: 1. La
\textbf{visualización en dos dimensiones} utilizando Análisis de
Componentes Principales (PCA).\\
2. La \textbf{tabla de centroides} (valores medios de cada variable en
cada cluster), reescalados a su escala original para facilitar la
interpretación.

\begin{Shaded}
\begin{Highlighting}[]
\CommentTok{\# Entrenamiento de K{-}means con k = 2}
\FunctionTok{set.seed}\NormalTok{(}\DecValTok{123}\NormalTok{)}
\NormalTok{km2 }\OtherTok{\textless{}{-}} \FunctionTok{kmeans}\NormalTok{(X\_scaled, }\AttributeTok{centers =} \DecValTok{2}\NormalTok{, }\AttributeTok{nstart =} \DecValTok{50}\NormalTok{, }\AttributeTok{iter.max =} \DecValTok{100}\NormalTok{)}

\CommentTok{\# Agregar la asignación de clusters al data frame original}
\NormalTok{Datos\_clustered }\OtherTok{\textless{}{-}}\NormalTok{ Datos }\SpecialCharTok{\%\textgreater{}\%}
  \FunctionTok{mutate}\NormalTok{(}\AttributeTok{cluster =} \FunctionTok{factor}\NormalTok{(km2}\SpecialCharTok{$}\NormalTok{cluster))}

\CommentTok{\# Visualización en PCA 2D}
\FunctionTok{fviz\_cluster}\NormalTok{(km2, }\AttributeTok{data =}\NormalTok{ X\_scaled,}
             \AttributeTok{geom =} \StringTok{"point"}\NormalTok{, }\AttributeTok{ellipse.type =} \StringTok{"norm"}\NormalTok{,}
             \AttributeTok{main =} \StringTok{"Visualización de clusters (k = 2)"}\NormalTok{) }
\end{Highlighting}
\end{Shaded}

\begin{center}\includegraphics{tarea_2_files/figure-latex/kmeans-exec-1} \end{center}

La gráfica de dispersión obtenida mediante \textbf{PCA en dos
dimensiones} muestra la distribución de los estudiantes en los dos
clusters formados por K-means.

Cada punto representa a un estudiante, y los colores distinguen los
clusters. Las elipses alrededor de cada grupo ilustran la dispersión y
solapamiento de las observaciones.

Se observa que: - Los dos clusters presentan cierta separación,
especialmente a lo largo de la primera componente principal (Dim1).\\
- Existe una zona de \textbf{superposición en el centro}, lo cual indica
que algunos estudiantes tienen características intermedias que no
permiten una diferenciación tajante.\\
- A pesar de ello, el modelo identifica \textbf{dos perfiles
predominantes}, que se analizarán en mayor detalle a través de los
centroides y la comparación de las variables numéricas y categóricas.

Para explorar mejor la separación entre grupos, se realiza una
\textbf{reducción de dimensionalidad mediante PCA a tres componentes} y
se grafica en 3D.\\
- Si el documento se renderiza a \textbf{HTML}, se muestra una
visualización \textbf{interactiva} con \texttt{plotly}.\\
- Si se renderiza a \textbf{PDF}, se muestra una \textbf{versión
estática} con \texttt{scatterplot3d}.

\begin{Shaded}
\begin{Highlighting}[]
\CommentTok{\# PCA a 3 componentes sobre los datos escalados}
\NormalTok{pca3 }\OtherTok{\textless{}{-}} \FunctionTok{prcomp}\NormalTok{(X\_scaled, }\AttributeTok{center =} \ConstantTok{FALSE}\NormalTok{, }\AttributeTok{scale. =} \ConstantTok{FALSE}\NormalTok{)}
\NormalTok{pc\_df }\OtherTok{\textless{}{-}} \FunctionTok{as.data.frame}\NormalTok{(pca3}\SpecialCharTok{$}\NormalTok{x[, }\DecValTok{1}\SpecialCharTok{:}\DecValTok{3}\NormalTok{])}
\FunctionTok{names}\NormalTok{(pc\_df) }\OtherTok{\textless{}{-}} \FunctionTok{c}\NormalTok{(}\StringTok{"PC1"}\NormalTok{, }\StringTok{"PC2"}\NormalTok{, }\StringTok{"PC3"}\NormalTok{)}
\NormalTok{pc\_df}\SpecialCharTok{$}\NormalTok{cluster }\OtherTok{\textless{}{-}} \FunctionTok{factor}\NormalTok{(km2}\SpecialCharTok{$}\NormalTok{cluster)}

\CommentTok{\# Etiquetas con varianza explicada para los ejes}
\NormalTok{var\_exp }\OtherTok{\textless{}{-}}\NormalTok{ (pca3}\SpecialCharTok{$}\NormalTok{sdev}\SpecialCharTok{\^{}}\DecValTok{2}\NormalTok{) }\SpecialCharTok{/} \FunctionTok{sum}\NormalTok{(pca3}\SpecialCharTok{$}\NormalTok{sdev}\SpecialCharTok{\^{}}\DecValTok{2}\NormalTok{)}
\NormalTok{xl }\OtherTok{\textless{}{-}} \FunctionTok{paste0}\NormalTok{(}\StringTok{"PC1 ("}\NormalTok{, }\FunctionTok{round}\NormalTok{(}\DecValTok{100} \SpecialCharTok{*}\NormalTok{ var\_exp[}\DecValTok{1}\NormalTok{], }\DecValTok{1}\NormalTok{), }\StringTok{"\%)"}\NormalTok{)}
\NormalTok{yl }\OtherTok{\textless{}{-}} \FunctionTok{paste0}\NormalTok{(}\StringTok{"PC2 ("}\NormalTok{, }\FunctionTok{round}\NormalTok{(}\DecValTok{100} \SpecialCharTok{*}\NormalTok{ var\_exp[}\DecValTok{2}\NormalTok{], }\DecValTok{1}\NormalTok{), }\StringTok{"\%)"}\NormalTok{)}
\NormalTok{zl }\OtherTok{\textless{}{-}} \FunctionTok{paste0}\NormalTok{(}\StringTok{"PC3 ("}\NormalTok{, }\FunctionTok{round}\NormalTok{(}\DecValTok{100} \SpecialCharTok{*}\NormalTok{ var\_exp[}\DecValTok{3}\NormalTok{], }\DecValTok{1}\NormalTok{), }\StringTok{"\%)"}\NormalTok{)}

\ControlFlowTok{if}\NormalTok{ (knitr}\SpecialCharTok{::}\FunctionTok{is\_html\_output}\NormalTok{()) \{}
  \CommentTok{\# {-}{-}{-} Interactivo (HTML) {-}{-}{-}}
  \CommentTok{\# install.packages("plotly") si hace falta}
  \FunctionTok{library}\NormalTok{(plotly)}
  \FunctionTok{plot\_ly}\NormalTok{(}
\NormalTok{    pc\_df,}
    \AttributeTok{x =} \SpecialCharTok{\textasciitilde{}}\NormalTok{PC1, }\AttributeTok{y =} \SpecialCharTok{\textasciitilde{}}\NormalTok{PC2, }\AttributeTok{z =} \SpecialCharTok{\textasciitilde{}}\NormalTok{PC3,}
    \AttributeTok{color =} \SpecialCharTok{\textasciitilde{}}\NormalTok{cluster, }\AttributeTok{symbol =} \SpecialCharTok{\textasciitilde{}}\NormalTok{cluster,}
    \AttributeTok{symbols =} \FunctionTok{c}\NormalTok{(}\StringTok{"circle"}\NormalTok{, }\StringTok{"triangle{-}up"}\NormalTok{),}
    \AttributeTok{type =} \StringTok{"scatter3d"}\NormalTok{, }\AttributeTok{mode =} \StringTok{"markers"}\NormalTok{,}
    \AttributeTok{marker =} \FunctionTok{list}\NormalTok{(}\AttributeTok{size =} \DecValTok{3}\NormalTok{)}
\NormalTok{  ) }\SpecialCharTok{\%\textgreater{}\%}
    \FunctionTok{layout}\NormalTok{(}
      \AttributeTok{title =} \StringTok{"Clusters en 3D (PCA)"}\NormalTok{,}
      \AttributeTok{scene =} \FunctionTok{list}\NormalTok{(}
        \AttributeTok{xaxis =} \FunctionTok{list}\NormalTok{(}\AttributeTok{title =}\NormalTok{ xl),}
        \AttributeTok{yaxis =} \FunctionTok{list}\NormalTok{(}\AttributeTok{title =}\NormalTok{ yl),}
        \AttributeTok{zaxis =} \FunctionTok{list}\NormalTok{(}\AttributeTok{title =}\NormalTok{ zl)}
\NormalTok{      ),}
      \AttributeTok{legend =} \FunctionTok{list}\NormalTok{(}\AttributeTok{title =} \FunctionTok{list}\NormalTok{(}\AttributeTok{text =} \StringTok{"Cluster"}\NormalTok{))}
\NormalTok{    )}
\NormalTok{\} }\ControlFlowTok{else}\NormalTok{ \{}
  \CommentTok{\# {-}{-}{-} Estático (PDF) {-}{-}{-}}
  \CommentTok{\# install.packages("scatterplot3d") si hace falta}
  \FunctionTok{library}\NormalTok{(scatterplot3d)}
\NormalTok{  cols }\OtherTok{\textless{}{-}} \FunctionTok{c}\NormalTok{(}\StringTok{"\#E64B35"}\NormalTok{, }\StringTok{"\#00A087"}\NormalTok{)  }\CommentTok{\# colores distintos por cluster}
\NormalTok{  pchs }\OtherTok{\textless{}{-}} \FunctionTok{c}\NormalTok{(}\DecValTok{16}\NormalTok{, }\DecValTok{17}\NormalTok{)}
\NormalTok{  col\_vec }\OtherTok{\textless{}{-}}\NormalTok{ cols[}\FunctionTok{as.integer}\NormalTok{(pc\_df}\SpecialCharTok{$}\NormalTok{cluster)]}
\NormalTok{  pch\_vec }\OtherTok{\textless{}{-}}\NormalTok{ pchs[}\FunctionTok{as.integer}\NormalTok{(pc\_df}\SpecialCharTok{$}\NormalTok{cluster)]}

  \FunctionTok{scatterplot3d}\NormalTok{(pc\_df}\SpecialCharTok{$}\NormalTok{PC1, pc\_df}\SpecialCharTok{$}\NormalTok{PC2, pc\_df}\SpecialCharTok{$}\NormalTok{PC3,}
                \AttributeTok{color =}\NormalTok{ col\_vec, }\AttributeTok{pch =}\NormalTok{ pch\_vec,}
                \AttributeTok{main =} \StringTok{"Clusters en 3D (PCA)"}\NormalTok{,}
                \AttributeTok{xlab =}\NormalTok{ xl, }\AttributeTok{ylab =}\NormalTok{ yl, }\AttributeTok{zlab =}\NormalTok{ zl,}
                \AttributeTok{grid =} \ConstantTok{TRUE}\NormalTok{, }\AttributeTok{box =} \ConstantTok{FALSE}\NormalTok{)}
  \FunctionTok{legend}\NormalTok{(}\StringTok{"topright"}\NormalTok{, }\AttributeTok{legend =} \FunctionTok{levels}\NormalTok{(pc\_df}\SpecialCharTok{$}\NormalTok{cluster),}
         \AttributeTok{col =}\NormalTok{ cols, }\AttributeTok{pch =}\NormalTok{ pchs, }\AttributeTok{bty =} \StringTok{"n"}\NormalTok{, }\AttributeTok{cex =} \FloatTok{0.9}\NormalTok{, }\AttributeTok{title =} \StringTok{"Cluster"}\NormalTok{)}
\NormalTok{\}}
\end{Highlighting}
\end{Shaded}

\begin{center}\includegraphics{tarea_2_files/figure-latex/pca-3d-1} \end{center}

En la gráfica se observa lo siguiente: - Los dos clusters se distinguen
principalmente a lo largo de la \textbf{primera componente (PC1)}.\\
- Existe un grado de \textbf{solapamiento} entre los grupos, consistente
con el valor moderado del índice de Silhouette.\\
- Aun con esta superposición, se aprecian \textbf{zonas de concentración
diferenciadas} que reflejan dos perfiles predominantes de estudiantes.

\subsection{Interpretación de los
clusters}\label{interpretaciuxf3n-de-los-clusters}

El objetivo de esta sección es \textbf{caracterizar los clusters}
formados por \textbf{K-means}. Para ello:

\begin{itemize}
\tightlist
\item
  Se comparan los \textbf{valores medios} de las \textbf{variables
  numéricas} en cada cluster.
\item
  Se analizan las \textbf{distribuciones de las variables categóricas}
  por cluster, con el fin de \textbf{enriquecer la interpretación} de
  los perfiles.
\end{itemize}

\subsubsection{Comparación de variables
numéricas.}\label{comparaciuxf3n-de-variables-numuxe9ricas.}

\begin{Shaded}
\begin{Highlighting}[]
\FunctionTok{library}\NormalTok{(dplyr)}

\CommentTok{\# Promedios por cluster}
\NormalTok{numeric\_means }\OtherTok{\textless{}{-}}\NormalTok{ Datos\_clustered }\SpecialCharTok{\%\textgreater{}\%}
  \FunctionTok{group\_by}\NormalTok{(cluster) }\SpecialCharTok{\%\textgreater{}\%}
  \FunctionTok{summarise}\NormalTok{(}\FunctionTok{across}\NormalTok{(}\FunctionTok{c}\NormalTok{(age, study\_hours\_per\_day, social\_media\_hours, netflix\_hours,}
\NormalTok{                     attendance\_percentage, sleep\_hours, exercise\_frequency,}
\NormalTok{                     mental\_health\_rating, exam\_score),}
\NormalTok{                   mean, }\AttributeTok{na.rm =} \ConstantTok{TRUE}\NormalTok{)) }\SpecialCharTok{\%\textgreater{}\%}
  \FunctionTok{ungroup}\NormalTok{()}

\NormalTok{numeric\_means}
\end{Highlighting}
\end{Shaded}

\begin{verbatim}
## # A tibble: 2 x 10
##   cluster   age study_hours_per_day social_media_hours netflix_hours
##   <fct>   <dbl>               <dbl>              <dbl>         <dbl>
## 1 1        20.5                2.55               2.67          1.99
## 2 2        20.5                4.47               2.36          1.66
## # i 5 more variables: attendance_percentage <dbl>, sleep_hours <dbl>,
## #   exercise_frequency <dbl>, mental_health_rating <dbl>, exam_score <dbl>
\end{verbatim}

\begin{itemize}
\item
  \textbf{Cluster 1}:\\
  Se caracteriza por \textbf{menor dedicación académica} (≈ 2.6 horas de
  estudio al día), acompañado de \textbf{mayor tiempo en redes sociales
  y Netflix}.\\
  Presenta una \textbf{menor asistencia a clases (83\%)}, menos
  ejercicio físico y \textbf{calificaciones más bajas en el examen final
  (≈ 55.6 puntos)}.\\
  Además, sus estudiantes reportan un \textbf{estado de salud mental más
  bajo}.
\item
  \textbf{Cluster 2}:\\
  Agrupa a los estudiantes con \textbf{hábitos más disciplinados}: en
  promedio estudian más de 4 horas al día, duermen y hacen ejercicio con
  mayor regularidad, y dedican menos tiempo a redes sociales y
  Netflix.\\
  También presentan \textbf{mayor asistencia (85\%)} y mejores
  indicadores de \textbf{salud mental}, lo que se refleja en un
  \textbf{mejor rendimiento académico (≈ 82.5 puntos)}.
\end{itemize}

Posteriormente, vemos la distribución de las variables categóricas que
conforman nuestros datos

\begin{Shaded}
\begin{Highlighting}[]
\FunctionTok{library}\NormalTok{(dplyr)}

\NormalTok{categorical\_summary }\OtherTok{\textless{}{-}}\NormalTok{ Datos\_clustered }\SpecialCharTok{\%\textgreater{}\%}
  \FunctionTok{group\_by}\NormalTok{(cluster) }\SpecialCharTok{\%\textgreater{}\%}
  \FunctionTok{summarise}\NormalTok{(}
    \AttributeTok{gender\_male\_pct =} \FunctionTok{mean}\NormalTok{(gender }\SpecialCharTok{==} \StringTok{"Male"}\NormalTok{) }\SpecialCharTok{*} \DecValTok{100}\NormalTok{,}
    \AttributeTok{part\_time\_job\_yes\_pct =} \FunctionTok{mean}\NormalTok{(part\_time\_job }\SpecialCharTok{==} \StringTok{"Yes"}\NormalTok{) }\SpecialCharTok{*} \DecValTok{100}\NormalTok{,}
    \AttributeTok{diet\_good\_pct =} \FunctionTok{mean}\NormalTok{(diet\_quality }\SpecialCharTok{==} \StringTok{"Good"}\NormalTok{) }\SpecialCharTok{*} \DecValTok{100}\NormalTok{,}
    \AttributeTok{extracurricular\_yes\_pct =} \FunctionTok{mean}\NormalTok{(extracurricular\_participation }\SpecialCharTok{==} \StringTok{"Yes"}\NormalTok{) }\SpecialCharTok{*} \DecValTok{100}\NormalTok{,}
    \AttributeTok{internet\_good\_pct =} \FunctionTok{mean}\NormalTok{(internet\_quality }\SpecialCharTok{==} \StringTok{"Good"}\NormalTok{) }\SpecialCharTok{*} \DecValTok{100}\NormalTok{,}
    \AttributeTok{.groups =} \StringTok{"drop"}
\NormalTok{  )}

\NormalTok{categorical\_summary}
\end{Highlighting}
\end{Shaded}

\begin{verbatim}
## # A tibble: 2 x 6
##   cluster gender_male_pct part_time_job_yes_pct diet_good_pct
##   <fct>             <dbl>                 <dbl>         <dbl>
## 1 1                  48.2                  23.0          37.0
## 2 2                  47.2                  20.2          38.6
## # i 2 more variables: extracurricular_yes_pct <dbl>, internet_good_pct <dbl>
\end{verbatim}

Las diferencias entre clusters en las variables categóricas resultan
menos marcadas que en las variables numéricas, aunque permiten observar
algunos matices:

\begin{itemize}
\tightlist
\item
  \textbf{Género:} la proporción de hombres es muy similar en ambos
  grupos (≈48\% en Cluster 1 vs 47\% en Cluster 2).\\
\item
  \textbf{Trabajo a tiempo parcial:} ligeramente más frecuente en el
  Cluster 1 (23\%) que en el Cluster 2 (20\%).\\
\item
  \textbf{Calidad de dieta:} algo mejor en el Cluster 2 (39\%) respecto
  al Cluster 1 (37\%).\\
\item
  \textbf{Participación extracurricular:} prácticamente idéntica en
  ambos grupos (≈32\%).\\
\item
  \textbf{Acceso a internet:} mayor en el Cluster 1 (47\%) en
  comparación con el Cluster 2 (42\%).
\end{itemize}

En conjunto, se confirma que las \textbf{principales diferencias entre
clusters están en los hábitos numéricos} (estudio, sueño, ocio,
ejercicio y rendimiento académico), mientras que las variables
categóricas aportan información complementaria pero con variaciones más
leves.

\end{document}
